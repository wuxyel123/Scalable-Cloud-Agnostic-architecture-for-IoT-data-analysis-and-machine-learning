\chapter{Technologies}
\label{cap:technologies}

\intro{In this chapter it's reported the study made on the various technologies taken into account to develop the project.}\\

\section{Cloud}
\label{cloud}
This section describes the services offered by \href{https://www.arubacloud.com/}{Aruba Cloud}\footcite{site:aruba-cloud}, \href{https://aws.amazon.com/it/}{Amazon Web Services}\footcite{site:aws} and \href{https://azure.microsoft.com/it-it/}{Microsoft Azure}\footcite{site:azure}, exploring their features, Pros: and Cons:. 
Services features are better described in the respective offical Aruba Cloud\footcite{site:aruba-docs}, AWS\footcite{site:aws-docs} and Azure\footcite{site:azure-docs} documentation, 
while Pros: and Cons: are based on the author's and the company experience as well as other users reviews that can be found in 
\href{https://www.trustradius.com/}{TrustRadius}\footcite{site:trust-radius} website.\\
These services are among the most used cloud services in the world, offering a wide range of services that can be used to develop the project.
It's also important to mention that these providers were chosen right away because of the already developed experience with them, both in the company and in the author of this thesis.

    \subsection{Aruba Cloud}
    \label{aruba-cloud}
    Aruba Cloud is a cloud service provider that offers a wide range of services like virtual machines, object storage, databases, and managed Kubernetes.\\

    \subsubsection{Bare Metal}
    \label{aruba-cloud:bare-metal}
    Aruba Cloud Bare Metal is a service that provides dedicated servers with high performance and reliability.\\
    \textbf{Pros:}
    \begin{itemize}
        \item High performance
        \item Reliable
        \item Cost effective
        \item Supports multiple operating systems
    \end{itemize}
    \textbf{Cons:}
    \begin{itemize}
        \item  Need to manage the whole infrastructure
        \item Not scalable
        \item Not upgradable
    \end{itemize}

    \subsubsection{Virtual Private server (VPS)}
    \label{aruba-cloud:vps}
    Aruba Cloud VPS is a service that provides virtual servers with high performance and reliability. It uses virtualizers such as OpenStacks, VMware and Hyper-V.\\
    \textbf{Pros:}
    \begin{itemize}
        \item High performance
        \item Reliable
        \item Cost effective
        \item Supports multiple operating systems
        \item Scalable
        \item SLA up to 99.95\%
    \end{itemize}
    \textbf{Cons:}
    \begin{itemize}
        \item Guaranteed resources only in the PRO plans
    \end{itemize}

    \subsubsection{Virtual Private cloud (VPC)}
    \label{aruba-cloud:vpc}
    Aruba Cloud VPC is a service that provides a virtual network isolated from the public network. It allows for the creation of multiple subnets and the configuration of security groups. It can be used to create a whole private cloud infrastructure.\\
    \textbf{Pros:}
    \begin{itemize}
        \item Isolated network
        \item Multiple subnets
        \item Security groups
        \item Scalable
        \item SLA up to 99.98\%
    \end{itemize}
    \textbf{Cons:}
    \begin{itemize}
        \item Complexity
        \item Costly
    \end{itemize}

    \subsubsection{Object Storage}
    \label{aruba-cloud:object-storage}
    Aruba Cloud Object Storage is a service that provides scalable and secure object storage. It can be used to store and retrieve large amounts of unstructured data.\\
    \textbf{Pros:}
    \begin{itemize}
        \item Scalable
        \item Secure
        \item Cost effective
        \item Highly available
        \item Reservable plans or pay per use
    \end{itemize}
    \textbf{Cons:}
    \begin{itemize}
        \item Limit for traffic to the public network
    \end{itemize}

    \subsubsection{Managed Kubernetes}
    \label{aruba-cloud:kubernetes}
    Aruba Cloud Managed Kubernetes is a service that provides a fully managed Kubernetes cluster. It can be used to deploy, scale, and manage containerized applications.\\
    \textbf{Pros:}
    \begin{itemize}
        \item Fully managed
        \item Scalable
        \item Secure
        \item Cost effective
        \item Maximum redundancy
        \item Minimum latency
        \item Rapid deployment
        \item Kubernetes native API
    \end{itemize}
    \textbf{Cons:}
    \begin{itemize}
        \item Complexity
    \end{itemize}


    \subsection{Amazon Web Services}

        \subsubsection{Batch}
        \label{aws:batch}
        AWS Batch is a fully managed service that enables developers to easily and efficiently run thousands of batch and machine learning computing jobs on AWS.\\
        \textbf{Pros:}
        \begin{itemize}
            \item Fully managed
            \item Scalable
            \item Cost effective
            \item Supports different batch processing scenarios
            \item Supports machine learning
            \item Easy to use
            \item Versatile
        \end{itemize}
        \textbf{Cons:}
        \begin{itemize}
            \item Not well documented
        \end{itemize}

        \subsubsection{Bedrock}
        \label{aws:bedrock}
        AWS Bedrock is a fully managed service that simplifies the deployment and management of machine learning models.\\
        Using AWS Bedrock users can chose from a variety of pre-trained models and deploy them on the edge.\\
        \textbf{Pros:}
        \begin{itemize}
            \item Fully managed
            \item Flexible
            \item Native support for Retrieval Augmented Generation (RAG) models
        \end{itemize}
        \textbf{Cons:}
        \begin{itemize}
            \item Costly
            \item New dependencies may introduce problems
            \item The chosen model may not be future proof
        \end{itemize}

        \subsubsection{DynamoDB}
        \label{aws:dynamodb}
        AWS DynamoDB is a fully managed NoSQL database service that provides fast and predictable performance with seamless scalability.
        Tables can store and retrieve virtually any amount of data, serving any level of request traffic.
        It automatically spreads the data and traffic for the table over a sufficient number of servers to handle the request capacity specified by the customer and the amount of data stored, while maintaining Cons:istent and fast performance.\\
        \textbf{Pros:}
        \begin{itemize}
            \item Fully managed
            \item Fast and predictable performance
            \item Scalable
            \item Highly available
            \item NoSQL
        \end{itemize}
        \textbf{Cons:}
        \begin{itemize}
            \item Hard to make changes against bulks of records
            \item Need to know at prior which queries will be made
        \end{itemize}

        \subsubsection{Elastic map reduce (EMR)}
        \label{aws:emr}
        AWS EMR is a big data platform that simplifies the deployment and management of big data frameworks, like Apache Hadoop and Apache Spark, on AWS.\\
        \textbf{Pros:}
        \begin{itemize}
            \item Fully managed
            \item Scalable
            \item Petabyte scale data processing
            \item Easy resources provisioning
            \item Reconfigurable
        \end{itemize}
        \textbf{Cons:}
        \begin{itemize}
            \item Complexity
            \item Costly
        \end{itemize}

        \subsubsection{Glue}
        \label{aws:glue}
        AWS Glue is a fully managed ETL service that enables efficient data integration on a large scale.\\
        \textbf{Pros:}
        \begin{itemize}
            \item Fully managed
            \item Pay per use
            \item Scalable
            \item Provides a centralize metadata repository
            \item Supports different data sources and formats
            \item Can automatically discover and catalog data from various sources
            \item Allow for job scheduling
            \item Data encryption
        \end{itemize}
        \textbf{Cons:}
        \begin{itemize}
            \item Costly for high workloads
            \item Performance issues with large datasets
            \item Complexity
        \end{itemize}

        \subsubsection{Greengrass}
        \label{aws:greengrass}
        AWS Greengrass is an open source edge runtime and cloud service used to build, deploy, and manage device software. 
        It enables the devices to process the data locally, while still using the cloud for management, analytics, and durable storage.\\
        It also enables encryption at rest and in transit and it can also extend device functionality with AWS Lambda functions.\\
        \textbf{Pros:}
        \begin{itemize}
            \item Edge computing
            \item Encryption at rest and in transit
            \item Extend device functionality with AWS Lambda functions
            \item ML models deployment
        \end{itemize}
        \textbf{Cons:}
        \begin{itemize}
            \item Restrained to AWS services
            \item Not platform agnostic
            \item Resource intensive for small devices
            \item Need a connection for the initial setup
        \end{itemize}
        
        \subsubsection{IoT Core}
        \label{aws:iot-core}
        AWS IoT Core is a fully managed cloud service that lets connected devices easily and securely interact with cloud applications and other devices.
        It is composed of multiple services like Device Management, Device Defender, Device Advisor, and IoT Analytics and only some of them can be used during the development.\\
        \textbf{Pros:}
        \begin{itemize}
            \item Composed of multiple services so only the necessary ones can be used
            \item Encription at rest and in transit
            \item Supports MQTT, HTTP, and WebSockets
            \item Allows for device management
            \item Allows for machine learning at edge
            \item Can trigger events thanks to custom rules
        \end{itemize}
        \textbf{Cons:}
        \begin{itemize}
            \item Not platform agnostic if installed on devices
            \item Lacks of integration for some devices
        \end{itemize}       

        \subsubsection{Kendra}
        \label{aws:kendra}
        AWS Kendra is a fully managed enterprise search service that allows developers to add search capabilities across various content repositories leveraging on built in connectors.\\
        \textbf{Pros:}
        \begin{itemize}
            \item Fully managed
            \item Scalable
            \item Supports multiple data sources
            \item Easy to use and set up
            \item Accurate search results
        \end{itemize}
        \textbf{Cons:}
        \begin{itemize}
            \item Costly
        \end{itemize}

        \subsubsection{Kinesis Data Firehose}
        \label{aws:kinesis-data-firehose}
        AWS Kinesis Data Firehose is a fully managed service that simplifies the process of capturing, transforming and loading streaming data.
        It acts as an ETL service that can capture, transform, and load streaming data into a variety of AWS services.
        Additionaly it can transform raw data in column oriented data formats like \href{https://parquet.apache.org/}{Apache Parquet}\footcite{site:apache-parquet}\\
        \textbf{Pros:}
        \begin{itemize}
            \item Fully managed
            \item Can read data from IoT core and Kinesis Data Streams
            \item Scalable
            \item Can transform data
            \item Can load data into different AWS services
            \item Supports batching based on time or size
        \end{itemize}
        \textbf{Cons:}
        \begin{itemize}
            \item Not always cost effective
            \item Limited transformation capabilities
            \item Does not support batching based on more complex rules
        \end{itemize}
        
        \subsubsection{Kinesis Data Streams} 
        \label{aws:kinesis-data-streams}
        AWS Kinesis Data Stream is a fully managed service that simplify the capture,
         processing and loading of streaming data in real time at any scale thus enabling real-time data analytics with ease.\\
        \textbf{Pros:}
        \begin{itemize}
            \item Fully managed
            \item Scalable
            \item Real-time and fast data processing
            \item Keeps data for 24 hours by default
        \end{itemize}
        \textbf{Cons:}
        \begin{itemize}
            \item Not always cost effective
            \item Limited data retention
            \item Limited data transformation
            \item Not useful for certain batch processing scenarios
        \end{itemize}

        \subsubsection{Lake Formation}
        \label{aws:lake-formation}
        AWS Lake Formation is a fully managed service that simplifies the creation, security and management of data lakes.
        It allows for cleaning and transforming the data using machine learning.\\
        \textbf{Pros:}
        \begin{itemize}
            \item Fully managed
            \item Scalable
            \item Secure
            \item Simplifies lake creation 
            \item Simplifies ingestion management    
            \item Simplifies permission management      
            \item Provides data auditing 
            \item Supports machine learning
            \item Supports data cataloging
        \end{itemize}
        \textbf{Cons:}
        \begin{itemize}
            \item Complexity
            \item Costly
            \item Not native support for all data sources
        \end{itemize}

        \subsubsection{Lambda}
        \label{aws:lambda}
        AWS Lambda is a serverless compute service that automatically manages the compute fleet, scaling precisely with the size of the workload. The key advantage of AWS Lambda is that it allows developers to run code without provisioning or managing servers, creating a highly scalable, flexible, and cost-effective environment for executing code. AWS Lambda supports a variety of programming languages and integrates seamlessly with other AWS services, making it a versatile tool for deploying microservices, building data processing workflows, and developing real-time applications. The service is designed to handle various use cases, from running simple, single-function applications to complex, multi-step workflows.\\
        \textbf{Pros:}
        \begin{itemize}
            \item Fully managed
            \item Serverless
            \item Pay per use
            \item Scalable
            \item Easy to integrate with other AWS services
            \item Supports multiple programming languages
            \item Easy to deploy and maintain
            \item Can run parallel executions
            \item Low time to market
            \item Supports custom libraries
        \end{itemize}
        \textbf{Cons:}
        \begin{itemize}
            \item Limited execution time (15 mins)
            \item Limited memory
            \item Limited environment variables
            \item Maximum 1000 concurrent executions
            \item Cold start
            \item Not cost effective for high workloads
        \end{itemize}

        \subsubsection{Managed Service for Apache Flink (MSF)}
        \label{aws:msf}
        AWS Managed Service for Apache Flink (MSF) is a fully managed service that simplifies the creation and execution of real-time applications using \href{https://flink.apache.org/}{Apache Flink}\footcite{site:apache-flink}, an open-source stream processing framework. This service provides developers with the tools to build sophisticated, high-throughput, low-latency data processing applications. MSF automates the underlying infrastructure management, enabling teams to focus on application logic rather than operational concerns. With built-in scalability, the service can handle varying workloads efficiently, making it suitable for both batch and stream processing. MSF also offers tight integration with other AWS services, providing a cohesive ecosystem for end-to-end data processing workflows.\\

        \textbf{Pros:}
        \begin{itemize}
            \item Fully managed
            \item Scalable
            \item Supports batch and stream processing
            \item Real-time processing
            \item Large-scale data processing
        \end{itemize}
        \textbf{Cons:}
        \begin{itemize}
            \item Complexity
        \end{itemize}

        \subsubsection{Managed Streaming for Kafka (MSK)}
        \label{aws:msk}
        AWS Managed Streaming for Apache Kafka (MSK) is a fully managed service that simplifies the setup, scaling, and management of \href{https://kafka.apache.org/}{Apache Kafka}\footcite{site:apache-kafka} clusters. Apache Kafka is a distributed streaming platform widely used for building real-time data pipelines and streaming applications. MSK allows developers to leverage Kafka's powerful capabilities without the operational burden of managing Kafka infrastructure. The service ensures high availability, durability, and security, integrating seamlessly with other AWS services. This enables users to build robust, scalable streaming solutions while focusing on application development rather than infrastructure management.\\
        \textbf{Pros:}
        \begin{itemize}
            \item Fully managed
            \item Scalable
            \item Cost effective
            \item Secure
            \item High availability
            \item Easy to integrate with other AWS services
        \end{itemize}
        \textbf{Cons:}
        \begin{itemize}
            \item Local testing challenges: hard to replicate the same environment in production and locally
            \item Not suitable for high traffic scenarios
            \item Complexity
        \end{itemize}

        \subsubsection{Sage Maker}
        \label{aws:sage-maker}
        AAWS SageMaker is a comprehensive cloud-based machine learning platform that enables developers and data scientists to build, train, and deploy machine learning models at scale. SageMaker provides a suite of tools that simplify each step of the machine learning workflow, from data labeling and preparation to model tuning and deployment. The platform supports a variety of machine learning frameworks and integrates with other AWS services, offering flexibility and interoperability. SageMaker's built-in algorithms and managed infrastructure allow users to focus on developing innovative models without worrying about the complexities of underlying hardware and software management.\\
        \textbf{Pros:}
        \begin{itemize}
            \item Fully managed
            \item Scalable
            \item Supports multiple machine learning frameworks
            \item Supports multiple programming languages
            \item Allow for easy model deployment
        \end{itemize}
        \textbf{Cons:}
        \begin{itemize}
            \item Cannot schedule training jobs
            \item Costly for high workloads
        \end{itemize}

        \subsubsection{Simple Storage Service (S3)} 
        \label{aws:s3}
        AWS S3 is an object storage service offering scalability, data availability, security, and performance.
        With S3, any amount of data can be stored and retrieved from anywhere on the web. 
        Data is stored as objects in buckets, with each object representing a file and its metadata.\\
        \textbf{Pros:}
        \begin{itemize}
            \item Scalable
            \item Highly available
            \item Secure
            \item Durable
            \item Cost effective
            \item No bucket size limit
            \item No limit to the number of objects that can be stored in a bucket
            \item Has different storage classes to fit frequent access, infrequent access, and long-term storage
        \end{itemize}
        \textbf{Cons:}
        \begin{itemize}
            \item Not suitable for small files
            \item Object size limit (5TB)
            \item Maximum 100 buckets per account
            \item Max 5GB per file upload via PUT operation
        \end{itemize}


    \subsection{Microsoft Azure}
    
        \subsubsection{Blob Storage}
        \label{azure:blob-storage}
        Azure Blob Storage is a fully managed object storage service that is highly scalable and available. 
        It can store large amounts of unstructured data, making it suitable for a wide range of workloads.\\
        \textbf{Pros:}
        \begin{itemize}
            \item Fully managed
            \item Scalable
            \item Highly available
            \item Secure
            \item Cost effective
            \item No limit to the number of objects that can be stored in a container
            \item Has different storage tiers to fit frequent access, infrequent access, and long-term storage
            \item Different storage options (Blob, archive, queue, file and disk) 
        \end{itemize}
        \textbf{Cons:}
        \begin{itemize}
            \item Not suitable for small files
            \item Object size limit (4TB)
            \item Maximum 2PB per account in US and Europe
            \item Maximum 500TB per account in other regions
        \end{itemize}

        \subsubsection{Cosmos DB}
        \label{azure:cosmos-db}
        Azure Cosmos DB is a fully managed NO-SQL database service supporting multiple data models. It supports multiple NoSQL databases like PostgreSQL, MongoDB, and Cassandra.\\
        \textbf{Pros:}
        \begin{itemize}
            \item Fully managed
            \item Scalable
            \item Multi model
            \item Global distribution
            \item Cons:istency levels based on the application needs
            \item Easy to use
        \end{itemize}
        \textbf{Cons:}
        \begin{itemize}
            \item Expensive
            \item Slow for complex queries
        \end{itemize}

        \subsubsection{DataBricks}
        \label{azure:databricks}
        Azure Databricks is a fully managed Apache Spark-based analytics platform supporting a variety of libraries and languages.\\
        \textbf{Pros:}
        \begin{itemize}
            \item Fully managed
            \item Scalable
            \item Supports multiple programming languages (Python, R, Scala, SQL, Java)
            \item Supports multiple libraries (Tensorflow, PyTorch, Scikit-learn, etc.)
            \item Open data lakehouse
        \end{itemize}
        \textbf{Cons:}
        \begin{itemize}
            \item Costly
            \item Complexity
            \item Hard to configure
        \end{itemize}

        \subsubsection{Data Explorer}
        \label{azure:data-explorer}
        Azure data explorer is a fully managed, real-time and high volume data analytics service. 
        It offers speed and low latency, being able to get quick insights from raw data.
        \textbf{Pros:}
        \begin{itemize}
            \item Fully managed
            \item Scalable
            \item Real-time data processing
            \item Low latency
            \item Supports multiple data sources
            \item Supports structured, semi-structured and unstructured data
            \item Fast data ingestion
            \item Can use batch processing
        \end{itemize}
        \textbf{Cons:}
        \begin{itemize}
            \item Complexity
            \item Costly
            \item Limited capabilities for data transformation
            \item Hard configuration
        \end{itemize}

        \subsubsection{Data Factory}
        \label{azure:data-factory}
        Azure data factory is a fully managed cloud-based data integration service.
        It provides tools to orchestrate data workflows while monitoring executions.\\
        \textbf{Pros:}
        \begin{itemize}
            \item Fully managed
            \item Scalable
            \item Can perfom data Analytics using Synapse
        \end{itemize}
        \textbf{Cons:}
        \begin{itemize}
            \item Complexity
            \item Limited transformation capabilities
        \end{itemize}

        \subsubsection{Data Lake Storage}
        \label{azure:data-lake-storage}
        Azure Data Lake Storage is a secure and scalable data lake platform. It provides a single place to store structured and unstructured data, making it easy to perform big data analytics.\\
        \textbf{Pros:}
        \begin{itemize}
            \item Fully managed
            \item Scalable
            \item Secure
            \item Cost effective
            \item Compatible with Hadoop
            \item Supports Python for data analytics
        \end{itemize}
        \textbf{Cons:}
        \begin{itemize}
            \item Data governance challenges
        \end{itemize}

        \subsubsection{Event Grid}
        \label{azure:event-grid}
        Azure Event Grid is a fully managed event routing service that simplifies the development of event-driven applications.\\
        \textbf{Pros:}
        \begin{itemize}
            \item Fully managed
            \item Scalable
            \item Supports MQTT5
            \item Supports multiple event sources
            \item Supports multiple event handlers
            \item Supports multiple event types
            \item Supports multiple programming languages
            \item Supports multiple event patterns
        \end{itemize}
        \textbf{Cons:}
        \begin{itemize}
            \item Complexity
            \item Limitations in event storage and retention
            \item Costly
        \end{itemize}

        \subsubsection{Event Hubs}
        \label{azure:event-hubs}
        Azure Event Hubs is a fully managed, real-time data ingestion service that is simple, secure, and scalable. It can be used to stream millions of events per second with low latency, from any source to any destination. It offers also native support for Apache Kafka, allowing user to run existing Kafka applications.\\
        \textbf{Pros:}
        \begin{itemize}
            \item Fully managed
            \item Scalable
            \item Secure
            \item Low latency
            \item Supports Apache Kafka
            \item Schema registry: centralize repository for schema management
            \item Real time data processing
        \end{itemize}
        \textbf{Cons:}
        \begin{itemize}
            \item Costly
            \item Complexity
            \item Limitation in event storage
            \item Cons:umers need to manage their state of processing
        \end{itemize}

        \subsubsection{Functions}
        \label{azure:functions}
        Azure functions is a serverless compute service that enables developers tu run code in response to 
        events without the need to manage the infrastructure.\\
        \textbf{Pros:}
        \begin{itemize}
            \item Fully managed
            \item Pay per use
            \item Scalable
            \item Supports multiple programming languages
            \item Easy to deploy and maintain
            \item Can run parallel executions
            \item Low time to market
            \item Supports custom libraries
        \end{itemize}
        \textbf{Cons:}
        \begin{itemize}
            \item Limited execution time (10 mins)
            \item Cold start
            \item Not cost effective for high workloads
        \end{itemize}

        \subsubsection{IoT Hub}
        \label{azure:iot-hub}
        Azure IoT Hub is a cloud service that serves as the bridge between IoT devices and solutions in the cloud, facilitating reliable and secure communication.
        It can handle and manage a large number of devices making it suitable both for small-scale and enterprise-level solutions.\\
        \textbf{Pros:}
        \begin{itemize}
            \item Secure
            \item Supports MQTT, AMQP, and HTTP
            \item Allows for device management
            \item Allows for machine learning at edge
            \item Can trigger events thanks to custom rules
            \item Can extend device functionality with Azure Functions
        \end{itemize}
        \textbf{Cons:}
        \begin{itemize}
            \item Not platform agnostic if installed on devices
            \item Lacks of integration for some devices
            \item Not well documented
            \item Costly
            \item Does not fully support MQTT5
        \end{itemize}

        \subsubsection{Machine Learning}
        \label{azure:machine-learning}
        Azure Machine Learning is a fully managed service that allows developers to build, train, and deploy machine learning models.\\
        \textbf{Pros:}
        \begin{itemize}
            \item Fully managed
            \item Scalable
            \item Supports multiple machine learning frameworks
            \item Supports multiple programming languages
            \item Allow for easy model deployment
            \item Cost effective
            \item Has MLOps capabilities
            \item Pay as you go
        \end{itemize}
        \textbf{Cons:}
        \begin{itemize}
            \item Cost raises when training big models
            \item Hard to optimize 
        \end{itemize}

\section{Present Solutions}
In this section are presented the solutions that are currently available on the market.

\subsection{Alleantia IoT Edge Hub}
\label{alleantia}
IoT Edge Hub is \href{www.alleantia.com}{Alleantia}\footcite{site:alleantia}'s plug and play solution for the industrial IoT.
It offers a wide range of features like device management, alarms and events, log management and report generation.
It also supports the integration with \href{https://www.microsoft.com/it-it/azure}{Microsoft Azure}\footcite{site:azure}.\\
\textbf{Pros:}
\begin{itemize}
    \item Plug and play
    \item Device management
    \item Alarms and events
    \item Log management
    \item Report generation
    \item Integration with Microsoft Azure
\end{itemize}
\textbf{Cons:}
\begin{itemize}
    \item Not platform agnostic
    \item Does not support \href{https://aws.amazon.com/it/}{Amazon Web Services}\footcite{site:aws}
\end{itemize}

\subsection{Eclipse Kura}
\label{kura}
\href{https://eclipse.dev/kura/}{Eclipse Kura}\footcite{site:kura} is an open source IoT Edge Framework that serves as a platform for building IoT gateways.
It's based on Java/OSGi and it provides API access to the hardware interfaces of IoT Gateways.\\
\textbf{Pros:}
\begin{itemize}
    \item Open source
    \item Platform agnostic
    \item Allows for flexible and modular development
    \item API access to hardware interfaces
    \item Introduces AI capabilities at the edge
\end{itemize}
\textbf{Cons:}  
\begin{itemize}
    \item Computational complexity for small devices
    \item Not well documented
    \item No native support for cloud services
\end{itemize}

\subsection{Eurotech Everyware Cloud}
\label{everyware-cloud}
\href{https://www.eurotech.com/}{Eurotech}\footcite{site:eurotech} Everyware Cloud is a cloud-based IoT Integration Platform with a microservices architecture that allows to connect, configure and manage IoT gateways and devices.\\
\textbf{Pros:}
\begin{itemize}
    \item Cloud-based
    \item Microservices architecture
    \item Allows to connect, configure and manage IoT gateways and devices
    \item Supports multiple protocols
    \item Supports multiple cloud services
\end{itemize}
\textbf{Cons:}
\begin{itemize}
    \item Last update in 2019
\end{itemize}

\subsection{STMicrelectronics X-Cube Cloud}
\label{stm}
\href{https://www.st.com/}{STMicrelectronics}\footcite{site:st-micro} X-Cube Cloud is a software package that enables the connection of STM32 microcontrollers to the cloud.\\
\textbf{Pros:}
\begin{itemize}
    \item Supports multiple cloud services
    \item Offers generic and secure connection to the cloud
    \item Supports multiple protocols
\end{itemize}
\textbf{Cons:}
\begin{itemize}
    \item Specific for STM32 microcontrollers
    \item Specific packages for each cloud service if you want to use the full potential
    \item The generic solution only runs on a subset of STM32 microcontrollers
\end{itemize}

\subsection{MQTTX}
\label{mqttx}
\href{https://mqttx.app/}{MQTTX}\footcite{site:mqttx} is a cross-platform MQTT 5.0 client tool that can be used to publish and subscribe to MQTT messages.\\
\textbf{Pros:}
\begin{itemize}
    \item Cross-platform
    \item Supports MQTT 5.0
    \item Connection management
    \item Log capabilities
    \item Data pipelines
    \item Device simulation
\end{itemize}
\textbf{Cons:}
\begin{itemize}
    \item MQTT client only
\end{itemize}

\subsection{EMQX}
\label{emqx}
\href{https://www.emqx.io/}{EMQX}\footcite{site:emqx} is an open source MQTT broker designed to be highly scalable.\\
\textbf{Pros:}
\begin{itemize}
    \item Open source
    \item Scalable
    \item Supports MQTT 5.0
    \item Supports web sockets
    \item Supports multiple cloud services
\end{itemize}
\textbf{Cons:}
\begin{itemize}
    \item The free version has limitations
\end{itemize}

\section{Machine Learning at edge} 
This section describes the technologies that can be used to build and deploy machine learning models at the edge and the Federated Learning approach.

\subsection{Tensorflow Lite}
\label{tensorflow-lite}
\href{https://www.tensorflow.org/lite}{Tensorflow Lite}\footcite{site:tflite} is the mobile and edge version of \href{https://www.tensorflow.org/}{Tensorflow}\footcite{site:tensorflow} that allows to run machine learning models on edge devices.\\
\textbf{Pros:}
\begin{itemize}
    \item Lightweight
    \item Supports multiple platforms
    \item Easy to use
    \item Well documented
\end{itemize}
\textbf{Cons:}
\begin{itemize}
    \item Need to pretrain the model before deploying it
    \item On-device training is limited to Unix-based systems
\end{itemize}

\subsection{Tiny Engine}
\label{tiny-engine}
\href{https://github.com/mit-han-lab/tinyengine}{Tiny Engine}\footcite{site:tinyengine}\footcite{lin2022ondevice} is a specialized machine learning framework designed to build, train, and deploy models on edge devices. It enables developers to convert TensorFlow Lite models into C++ code, facilitating efficient deployment on resource-constrained devices. Tiny Engine supports multiple platforms and utilizes pre-trained models, making it a versatile tool for various edge computing applications. The framework is lightweight, ensuring minimal resource consumption and fast inference times, which are critical for real-time, on-device machine learning tasks. Tiny Engine is particularly suited for applications in areas such as IoT, where low power consumption and quick response times are essential.

\textbf{Pros:}
\begin{itemize}
    \item Lightweight
    \item Supports multiple platforms
    \item Converts Tensorflow Lite models to C++
    \item Uses pre trained models
\end{itemize}
\textbf{Cons:}
\begin{itemize}
    \item Need to pretrain the model before deploying it
    \item Better support for MCUnet\footcite{lin2020mcunet}
    \item Not well documented for deployment of custom models
\end{itemize}



\subsection{Federated Learning and Transfer Learning}
Federated Learning is a machine learning approach that trains an algorithm across multiple decentralized edge devices or servers holding local data samples, without exchanging them. This approach is advantageous because it allows for privacy preservation and data security, minimizing the risk of sensitive information being exposed. Federated Learning also makes it feasible to train models on devices with low computational power, which is particularly useful in edge computing environments where computational resources are limited. As described in \enquote{EdgeFed: Optimized Federated Learning Based on Edge Computing}\footcite{9260194}, this method enables the development of sophisticated machine learning models by utilizing the collective power of multiple devices, enhancing both the efficiency and effectiveness of the training process.
\\
Another important approach is Transfer Learning, a technique that transfers the knowledge from a model trained on a specific task to a new, related task. This approach significantly improves the performance of the model on the new task and reduces the time and resources needed for training. Transfer Learning is especially valuable when there is limited data available for the new task, as it leverages the pre-existing knowledge embedded in the model. As detailed in \enquote{Federated learning for IoT devices: Enhancing TinyML with on-board training}\footcite{FICCO2024102189}, this method can enhance the capabilities of IoT devices by enabling them to perform complex tasks with improved accuracy and efficiency, without the need for extensive retraining.

\textbf{Pros:}
\begin{itemize}
    \item Privacy preservation
    \item Data security
    \item No need for data centralization
    \item Low computational power needed
    \item Low latency
\end{itemize}
\textbf{Cons:}
\begin{itemize}
    \item Complex to implement
    \item Limited capabilities
\end{itemize}