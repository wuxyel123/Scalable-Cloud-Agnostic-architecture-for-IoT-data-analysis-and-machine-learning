\chapter{Conclusion}
\label{cap:conclusion}

\intro{In this chapter, the conclusions of the work are presented. The objectives achieved are discussed, as well as the future developments of the project. The author also reflects on what was learned during the development of the project.}\\

\section{Objectives achieved}
The main goal of the project, as described in \ref{sec:objectives-and-requirements} was to develop a scalable cloud-agnostic architecture able to ingest data from multiple sources, process it and store it in a data lake. The architecture was successfully engineered, developed and tested, as described in \ref{cap:method}. The sample architecture has been tested both with \href{https://www.arubacloud.com/}{Aruba Cloud}\footcite{site:aruba-cloud} and \href{https://azure.microsoft.com/it-it/}{Microsoft Azure}\footcite{site:azure} to assess its cloud-agnostic nature.
In chapter \ref{cap:real_implementation} a real-world implementation of the base architecture was presented, showing the flexibility and scalability of the solution. The security constraints were also taken into account and obtained thanks to the shared responsibility model of the cloud providers and the use of encryption at rest and in transit.\\

\section{Future developments}
The future development of the project can be focused both on the base architecture described in chapter \ref{cap:method} and on the real-world implementation described in chapter \ref{cap:real_implementation}.

\subsection{Base architecture}
For the base architecture, the future developments can be focused on the following points:
\begin{itemize}
    \item \textbf{Edge Computing}: the architecture can be extended to support edge computing. A useful feature would be to be able to train machine learning models on the cloud and deploy them on edge devices.
    \item \textbf{Test on other cloud providers}: the architecture can be tested on other cloud providers to assess its cloud-agnostic nature.
\end{itemize}



\subsection{Real World Implementation}

The real-world implementation can benefit a lot more from future developments since it's a real-world use case of the base architecture. The future developments can be focused on the following points:

\subsubsection{Data Ingestion}
\begin{itemize}
    \item \textbf{Mobile Application Integration}:
        \begin{itemize}
            \item Develop a mobile application for data data management and analytics visualization.
            \item Ensure secure data transfer protocols are implemented.
        \end{itemize}
    \item \textbf{Dedicated Hardware}:
        \begin{itemize}
            \item Implement dedicated hardware for direct data upload from suits to the cloud.
        \end{itemize}
\end{itemize}

\subsubsection{Trigger Analytics Pipeline}
\begin{itemize}
    \item \textbf{Event-Driven Services}:
        \begin{itemize}
            \item Implement event-driven services (e.g., cloud functions or serverless architecture) to trigger the analytics pipeline upon data upload.
        \end{itemize}
\end{itemize}

\subsubsection{Data Processing}
\begin{itemize}
    \item \textbf{Advanced Data Preprocessing}:
        \begin{itemize}
            \item Enhance the preprocessing scripts to handle more complex data cleaning and transformation tasks.
        \end{itemize}
    \item \textbf{Enhanced Analytics Algorithms}:
        \begin{itemize}
            \item Develop more sophisticated algorithms for analyzing preprocessed data and extracting detailed insights related to rider performance.
        \end{itemize}
    \item \textbf{Crash Detection Algorithms}:
        \begin{itemize}
            \item Develop specialized algorithms to detect crash events with higher accuracy.
        \end{itemize}
\end{itemize}

\subsubsection{User Access and Notification}
\begin{itemize}
    \item \textbf{User Notification System}:
        \begin{itemize}
            \item Implement push notifications or WebSocket notifications to inform users when their analytics data is ready.
        \end{itemize}
\end{itemize}

\subsubsection{Data Retention and Management}
\begin{itemize}
    \item \textbf{User Download Window}:
        \begin{itemize}
            \item Provide users with a time window to download and save their raw data before it is deleted.
        \end{itemize}
\end{itemize}

\subsubsection{Technical Detail and Tools}
\begin{itemize}
    \item \textbf{Security Enhancements}:
        \begin{itemize}
            \item Enhance security features for API Gateway (e.g., advanced authentication, encryption).
        \end{itemize}
\end{itemize}

\subsubsection{Testing}
\begin{itemize}
    \item \textbf{Automated Unit Testing Framework}:
        \begin{itemize}
            \item Implement an automated unit testing framework to ensure all components are thoroughly tested.
        \end{itemize}
    \item \textbf{Comprehensive Integration Testing}:
        \begin{itemize}
            \item Perform extensive integration testing to validate interactions between all system components.
        \end{itemize}
\end{itemize}

\subsubsection{Additional Enhancements}
\begin{itemize}
    \item \textbf{Performance Optimization}:
        \begin{itemize}
            \item Optimize the performance of the analytics pipeline.
            \item Conduct load testing to ensure the system can handle high data volumes and concurrent users.
        \end{itemize}
    \item \textbf{Scalability Improvements}:
        \begin{itemize}
            \item Enhance the system architecture to support future growth and increased data volume.
            \item Deploy the system on a containerized platform for improved scalability and resource utilization.
        \end{itemize}
\end{itemize}

\section{Final considerations}

The development of this project has been a great learning experience. I gained valuable insights into cloud computing, big data processing, and data analytics. The project has provided a hands-on opportunity to work with cutting-edge technologies and tools and to apply theoretical knowledge to real-world scenarios. I have also learned about the challenges and complexities involved in designing and implementing a scalable and secure data processing system. The project has reinforced the importance of proper planning, design, and testing in software development, and has highlighted the need for continuous learning and adaptation in the fast-paced field of technology. Overall, the project has been a rewarding experience that has enriched my skills and knowledge in cloud computing and architecture design. I am looking forward to applying these learnings in future projects and endeavours.
