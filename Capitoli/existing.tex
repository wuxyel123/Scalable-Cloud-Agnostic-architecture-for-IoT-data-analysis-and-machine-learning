\chapter{Analysis of existing solutions}
\label{cap:existing}
In this chapter, is provided an overview of the solutions currently available on the market that could be integrated into this project. These solutions include various softwares and tools that could be leveraged for building a scalable, cloud-agnostic architecture for IoT data analysis and machine learning applications. The chapter focuses on evaluating these technologies in terms of their compatibility, scalability, and potential for integration with various cloud providers.


\section{Alleantia IoT Edge Hub}
\label{alleantia}
IoT Edge Hub is Alleantia's\cite{site:alleantia} plug-and-play solution for IoT. It allows for quick integration with IoT systems but has limitations in terms of platform support.

\textbf{Pros:}
\begin{itemize}
    \item Plug and play
    \item Device management: enables remote management and updates of devices
    \item Alarms and events handling
    \item Log management
    \item Report generation
    \item Integration with Microsoft Azure
\end{itemize}

\textbf{Cons:}
\begin{itemize}
    \item Platform dependent (Microsoft Azure)
    \item Does not support Amazon Web Services (AWS)
\end{itemize}

\section{Eclipse Kura}
\label{kura}
Eclipse Kura\cite{site:kura} is an open-source IoT Edge Framework for building IoT gateways. It is based on Java and the OSGi\cite{site:osgi} framework, which enhances Java’s modular programming capabilities. It provides API access to hardware interfaces of IoT gateways (physical devices that enable sensors to communicate with the internet).

\textbf{Pros:}
\begin{itemize}
    \item Open source: allows for customization
    \item Platform agnostic
    \item Enables flexible and modular development
    \item Provides API access to hardware interfaces
    \item Supports AI capabilities at the edge
\end{itemize}

\textbf{Cons:}
\begin{itemize}
    \item High computational complexity for less powerful devices
    \item Lacks of documentation, making it difficult to learn
\end{itemize}

\section{Eurotech Everyware Cloud}
\label{everyware-cloud}
Eurotech\cite{site:eurotech} Everywhere Cloud is a cloud-based platform offered by Eurotech, a global company specializing in IoT solutions. The Everywhere Cloud is designed to provide end-to-end IoT services, enabling seamless integration and management of connected devices, data processing, and application deployment. It serves as a comprehensive platform that simplifies IoT deployments by offering device connectivity, data collection, storage, and real-time analytics.

\textbf{Pros:}
\begin{itemize}
    \item Cloud-based
    \item Allows for connection, configuration, and management of IoT gateways and devices
    \item Supports multiple protocols
    \item Supports multiple cloud providers (AWS and Azure)
\end{itemize}

\textbf{Cons:}
\begin{itemize}
    \item Last updated in 2019, which could mean it is outdated compared to newer cloud services
\end{itemize}

\section{STMicroelectronics X-Cube Cloud}
\label{stm}
STMicroelectronics\cite{site:st-micro} X-Cube Cloud is a software package designed to connect STM32 microcontrollers to the cloud. STM32 microcontrollers are widely used in IoT applications, and this software package helps in securely linking them to various cloud providers.

The X-Cube Cloud package is available in two main versions: a generic version and dedicated versions tailored to individual cloud providers. Each of these versions offers different levels of functionality and compatibility.

\subsection*{Generic Version}
The \textbf{generic version} of X-Cube Cloud provides a cloud-agnostic solution, enabling basic cloud connectivity across multiple platforms. However, it is limited in its features and supports only a subset of STM32 microcontrollers.

\textbf{Pros:}
\begin{itemize}
    \item Supports multiple cloud providers
    \item Provides essential cloud connectivity and communication protocols like MQTT and HTTP
    \item Works with a range of cloud platforms without requiring specific versions
\end{itemize}

\textbf{Cons:}
\begin{itemize}
    \item Limited feature set, offering only basic cloud functionalities
    \item Supports only a subset of STM32 microcontrollers
    \item Lacks advanced cloud-specific optimizations, such as OTA updates or cloud-native device management
\end{itemize}

\subsection*{Dedicated Versions}
The \textbf{dedicated versions} of X-Cube Cloud are specifically developed for particular cloud providers, such as AWS or Microsoft Azure. These versions are optimized to provide full integration with the respective cloud services, offering a richer feature set and broader microcontroller compatibility.

\textbf{Pros:}
\begin{itemize}
    \item Full integration with specific cloud platforms, providing advanced cloud features
    \item Optimized for better performance, security, and scalability on each cloud provider
    \item Supports a wider range of STM32 microcontrollers
\end{itemize}

\textbf{Cons:}
\begin{itemize}
    \item A separate version is required for each cloud provider to access full functionalities
    \item Not as flexible as the generic version for switching between multiple cloud providers
\end{itemize}

\subsection*{Comparison of Versions}
In summary, the \textbf{generic version} provides flexibility in working with multiple cloud providers but comes with limitations in features and device support. On the other hand, the \textbf{dedicated versions} offer full-featured cloud integration and broader compatibility but require a separate software version for each cloud provider to unlock all features.


\section{MQTTX}
\label{mqttx}
MQTTX\cite{site:mqttx} is a cross-platform MQTT 5.0 client that facilitates publishing and subscribing to MQTT messages, testing connections, and simulating IoT devices. It provides a user-friendly interface for managing multiple connections to MQTT brokers, making it an essential tool for testing and developing MQTT-based IoT applications.
MQTTX also allows users to create virtual devices that can simulate real-world IoT devices. These virtual devices can publish and subscribe to MQTT topics, simulating the behavior of actual IoT hardware. Data pipelines is another useful feature in MQTTX. It refers to the ability to process and route incoming or outgoing MQTT messages through a defined workflow. This allows users to simulate complex data flows by linking multiple topics and processing the data as it moves through different stages, such as filtering, transforming, or rerouting the data.

\textbf{Pros:}
\begin{itemize}
    \item Connection management: Easily manage multiple connections to different MQTT brokers simultaneously.
    \item Logging capabilities: Provides detailed logs of MQTT transactions for better analysis and debugging.
    \item Data pipelines: Supports message processing and routing through custom workflows, allowing for complex testing scenarios.
    \item Device simulation capabilities: Can create virtual devices to publish and subscribe to topics, simulating real-world IoT devices for comprehensive testing.
\end{itemize}

\textbf{Cons:}
\begin{itemize}
    \item \Poorly documented data pipelines: The feature is not well-explained, making it difficult for users to create and debug pipelines effectively, especially in complex scenarios.
\end{itemize}


\section{EMQX}
\label{emqx}
EMQX\cite{site:emqx} is an open-source MQTT broker that is designed to be highly scalable. It supports various messaging protocols and cloud services, making it a robust option for IoT systems requiring high scalability.

\textbf{Pros:}
\begin{itemize}
    \item Supports MQTT 5.0
    \item Compatible with WebSockets
    \item Supports multiple cloud services, including AWS and Azure
\end{itemize}

\textbf{Cons:}
\begin{itemize}
    \item Though open source, access to all features requires purchasing the enterprise version
\end{itemize}

\section{Machine learning at edge} 
This section describes the technologies that can be used to build and deploy machine learning models at the edge and the Federated Learning approach. Even if they have not yet been implemented in the solution proposed in chapter \ref{cap:method}, they could still be considered for future development.

\subsection*{Tensorflow Lite}
\label{tensorflow-lite}
Tensorflow Lite\cite{site:tflite} is the mobile and edge version of Tensorflow\cite{site:tensorflow} that allows to run machine learning models on edge devices.\\
\textbf{Pros:}
\begin{itemize}
    \item Lightweight
    \item Supports multiple operating systems both mobile and edge
    \item Easy to use
    \item Well documented
\end{itemize}
\textbf{Cons:}
\begin{itemize}
    \item On-device training is limited to Unix-based systems, only inference is supported on edge devices
\end{itemize}

\subsection*{Tiny Engine}
\label{tiny-engine}
Tiny Engine\cite{site:tinyengine}\cite{lin2022ondevice} is a specialized machine learning framework designed to build, train, and deploy models on edge devices. Tiny Engine can utilize pre-trained models, making it a versatile tool for various edge computing applications. The framework is lightweight, ensuring minimal resource consumption and fast inference times, which are critical for real-time, on-device machine learning tasks. Tiny Engine is particularly suited for applications in areas such as IoT, where low power consumption and quick response times are essential.

\textbf{Pros:}
\begin{itemize}
    \item Supports multiple platforms
    \item Allows to convert TensorFlow Lite models to C++
    \item Easy deployment of pre-trained models
\end{itemize}
\textbf{Cons:}
\begin{itemize}
    \item Need to pre-train the model before deploying it
    \item MCUnet\cite{lin2020mcunet} (An algorithm for deep learning on microcontrollers) is the only well-documented model
    \item Custom models are hard to deploy and there is a lack of documentation
\end{itemize}



\subsection*{Federated Learning and Transfer Learning}
Federated Learning is a machine learning approach that trains an algorithm across multiple decentralized edge devices or servers holding local data samples, without exchanging them. This approach is advantageous because it allows for privacy preservation and data security, minimizing the risk of sensitive information being exposed. Federated Learning also makes it feasible to train models on devices with low computational power, which is particularly useful in edge computing environments where computational resources are limited. As described in \cite{9260194}, this method enables the development of sophisticated machine learning models by utilizing the collective power of multiple devices, enhancing both the efficiency and effectiveness of the training process.
\\
Another important approach is Transfer Learning, a technique that transfers the knowledge from a model trained on a specific task to a new, related task. This approach significantly improves the performance of the model on the new task and reduces the time and resources needed for training. Transfer Learning is especially valuable when there is limited data available for the new task, as it leverages the pre-existing knowledge embedded in the model. As detailed in \enquote{Federated learning for IoT devices: Enhancing TinyML with on-board training}\cite{FICCO2024102189}, this method can enhance the capabilities of IoT devices by enabling them to perform complex tasks with improved accuracy and efficiency, without the need for extensive retraining.

\textbf{Pros:}
\begin{itemize}
    \item Privacy preservation
    \item Data security
    \item No need for data centralization
    \item Low computational power needed
    \item Predictions made on the edge reduce latency
\end{itemize}
\textbf{Cons:}
\begin{itemize}
    \item Complicated to implement
    \item Limited capabilities compared to centralized training
\end{itemize}


\section{Comparison of existing technologies}
After analyzing various solutions available on the market, it is clear that each technology has its own strengths and weaknesses, depending on the use case. Below is a comparison of these technologies in terms of their scalability, platform dependency, cloud integration capabilities, and potential use for machine learning at the edge.

\begin{itemize}
    \item \textbf{Alleantia IoT Edge Hub} is a powerful industrial IoT solution but lacks cloud-agnostic capabilities, as it only supports Microsoft Azure. This makes it unsuitable for a multi-cloud environment and limits its flexibility in cloud integration.

    \item \textbf{Eclipse Kura} is open-source and platform-agnostic, providing flexibility and modularity. However, it suffers from high computational requirements and poor documentation, leading to a steep learning curve, which could be a barrier to effective implementation, especially when considering advanced features like machine learning at the edge.

    \item \textbf{Eurotech Everyware Cloud} offers robust cloud integration but has not been updated since 2019. This raises concerns about long-term support and compatibility with newer cloud and edge technologies, including the growing demand for edge-based machine learning solutions.

    \item \textbf{STMicroelectronics X-Cube Cloud} is a specialized solution tailored to STM32 microcontrollers, but its platform-specific nature limits its general applicability in broader cloud-agnostic architectures. Although powerful for edge computing, it does not offer the flexibility needed for cross-platform machine learning deployment, making it less suitable for scaling machine learning models across different devices and environments.

    \item \textbf{MQTTX} is useful for testing MQTT connections and simulating IoT devices, but it lacks the robust features required for production environments and large-scale cloud integration, limiting its utility in complex IoT ecosystems that include edge-based machine learning deployments.

    \item \textbf{EMQX} stands out as a highly scalable, cloud-agnostic MQTT broker that supports multiple cloud platforms, making it an excellent choice for a scalable IoT architecture. Its support for MQTT 5.0 and ability to integrate seamlessly with cloud services make it ideal for handling large IoT data streams. Additionally, it is well-suited for real-time data processing at the edge, which can facilitate machine learning deployments on edge devices. While the enterprise version is paid, the open-source version offers sufficient features for scalable IoT and potential machine learning integration.

    \item \textbf{Machine Learning at the Edge Technologies} such as \textbf{Tensorflow Lite} and \textbf{Tiny Engine} provide the ability to run machine learning models directly on edge devices. Tensorflow Lite is lightweight, well-documented, and supports multiple operating systems, making it suitable for a range of IoT applications. Tiny Engine, while offering specialized support for edge devices, has limited support for custom models and requires pre-trained models for deployment. Both frameworks could significantly enhance IoT solutions by enabling low-latency, on-device inference, which is crucial for real-time applications. However, implementing these technologies would require robust device management and efficient data handling to ensure scalability across the architecture.
\end{itemize}

\section{Chosen solution}
\label{sec:chosen-solution}
After evaluating all available options, \textbf{EMQX} was chosen as the messaging protocol broker for the architecture due to its strong cloud-agnostic features, scalability, and support for multiple cloud services. While Eclipse Kura was considered for its flexibility and open-source nature, its high computational requirements and poor documentation posed significant challenges, especially for rapid deployment. Other options like Alleantia and STMicroelectronics X-Cube Cloud were excluded because they were platform-dependent or limited to specific hardware, which goes against the core requirement for a cloud-agnostic and scalable architecture.

\textbf{EMQX} provides native support for MQTT 5.0, scalability across various cloud providers, and strong open-source community support, making it ideal for implementing a scalable IoT architecture. This architecture can ingest and analyze data from IoT devices while remaining cloud-agnostic, ensuring flexibility and ease of future expansion.

In terms of \textbf{machine learning at the edge}, technologies such as Tensorflow Lite and Tiny Engine were not immediately implemented in the current solution but could be integrated into future iterations. Once the architecture has been fully developed and tested for scalability and efficiency, adding machine learning at the edge would provide additional benefits such as real-time data processing, reduced latency, and enhanced analytics capabilities. Federated Learning and Transfer Learning approaches would also enable more advanced model training directly on edge devices, ensuring privacy and reducing the need for centralized data processing. This would be particularly valuable in IoT environments where devices have limited computational power but require sophisticated analytics for real-time decision-making.
