  \chapter{Conclusion}
\label{cap:conclusion}

In this chapter, the conclusions of the work are presented. The objectives achieved are discussed, as well as the future developments of the project. The author also reflects on the learnings gained during the project’s development.

\section{Objectives achieved}

The main goal of this project was to design and implement a scalable, cloud-agnostic architecture capable of ingesting data from multiple sources, processing it, and storing it in a data lake. This objective was successfully achieved through the engineering, development, and testing of the architecture, as described in Chapter \ref{cap:method}. The system was tested on both Aruba Cloud and Microsoft Azure to assess its cloud-agnostic nature, confirming its capability to function seamlessly across different cloud platforms.

The architecture was validated through a real-world implementation, which demonstrated its flexibility and scalability (Chapter \ref{cap:real_implementation}). The system was capable of ingesting data, processing analytics, and handling large volumes of information in a cloud-agnostic manner, ensuring that the architecture could scale effectively to meet real-world demands. Security constraints were also addressed using the shared responsibility model of the cloud providers, and encryption was implemented both at rest and in transit.

\section{Future developments}

The future development of this project can focus primarily on enhancing the base architecture. The Proof of Concept (PoC) was designed to validate its capabilities, and moving forward, there are several areas for improvement and extension that can solidify the architecture’s scalability, flexibility, and cloud-agnostic nature.

\subsection{Architecture enhancements}
To further strengthen the base architecture, the following advancements are recommended:

\begin{itemize}
    \item \textbf{Edge Computing Integration}: Incorporating edge computing capabilities into the architecture would significantly reduce latency and improve real-time processing. This can be achieved by training machine learning models in the cloud and deploying them on edge devices. With this approach, data can be processed closer to its source, minimizing the need for constant communication with the cloud and allowing for faster decision-making, which is especially valuable for time-sensitive applications like IoT and autonomous systems.
    
    \item \textbf{Wider Cloud Provider Testing}: While the current implementation demonstrated the architecture’s cloud-agnostic capabilities across Aruba Cloud and Azure, future testing should expand to include other major cloud platforms such as AWS, Google Cloud, and Oracle Cloud. This will help validate that the architecture can seamlessly adapt to different environments without major modifications, ensuring compatibility across a broader range of platforms and reducing the risk of vendor lock-in.

    \item \textbf{Enhanced Security and Compliance}: As the architecture grows in complexity and scale, incorporating advanced security features will be essential. This includes integrating role-based access control (RBAC), multi-factor authentication (MFA), and enhanced encryption standards. Additionally, ensuring compliance with industry regulations such as GDPR, HIPAA, and SOC2 will become increasingly important as the system scales and handles more sensitive data.

    \item \textbf{Performance Optimization}: Optimizing the performance of the data processing pipeline, particularly for handling high volumes of concurrent requests, should be a key focus. This can involve fine-tuning data transfer mechanisms, improving parallel processing capabilities, and leveraging distributed computing techniques. Additionally, containerizing the application using platforms like Kubernetes would improve scalability and resource management, allowing the architecture to handle larger workloads efficiently.

    \item \textbf{Comprehensive Testing Framework}: Introducing an automated testing framework for unit, integration, and performance testing will help ensure that future developments are thoroughly validated before deployment. Automated testing will streamline the development cycle, making it easier to detect and resolve issues early while maintaining system stability.
\end{itemize}

\section{Final thoughts}

The development of a scalable, cloud-agnostic architecture provided valuable insights into cloud computing, data ingestion, and real-time processing. The PoC tested the architecture's flexibility, reliability, and scalability across multiple platforms, laying the groundwork for future enhancements. By focusing on edge computing, further cloud-agnostic testing, and performance optimization, this architecture has the potential to support a wide range of use cases beyond its initial scope.

Building adaptable solutions that are cloud-agnostic ensures the architecture’s long-term sustainability and allows it to evolve with industry needs. As organizations increasingly require scalable, secure, and flexible cloud architectures, this project provides a robust foundation for meeting those demands across diverse applications and environments.
